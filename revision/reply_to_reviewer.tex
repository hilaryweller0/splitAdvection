\documentclass[12pt]{article}
\usepackage[margin=2cm]{geometry}
\usepackage{url,times}
\usepackage{amsmath}
\usepackage{graphicx}
\usepackage[authoryear]{natbib}
\usepackage{todonotes}

\begin{document}

\title{Responses to reviewer comments on artile QJ-17-0017\\
Dimension Splitting and a Long Time-Step Multi-Dimensional Scheme for Atmospheric Transport}
\author{Yumeng Chen \and Hilary Weller \and Stephen Pring \and and James Shaw}
\maketitle

\section*{Responses to reviewer 1}

\begin{description}

\item [Comment]:
I like the subject matter, and we are certainly in need of new ways of gaining efficiency for transport algorithms and for full dynamics for that matter. I think dimensionally split semi-Lagrangian schemes could be very competitive, but I have concerns about their accuracy on the cubed-sphere (as do the authors). I like the ingenuity of the test cases as well. I have three major concerns regarding the manuscript though. I have annotated these and other concerns directly to the PDF document, which is attached to this review. I think the content is important and needed and that the manuscript should be published, but subject to what I think amount to major revisions.

\item [Comment 1]: This is the most minor, but there is a study for semi-Lagrangian dimensionally split transport on the cubed-sphere that needs to be considered and discussed: \cite{GNQ14}. Additionally, \cite{KNK15} was cited as a dimensionally split scheme, but it is not. The reconstruction is performed in dimensional sweeps, but the scheme itself is multi-dimensional.

\item [Response]: 
Thank-you for pointing out the \cite{GNQ14} paper. See response to pdf annotations below.

\item [Comment 2]:
It is stated that results in this 2-D framework will generalize well to the cubed-sphere, but I believe that in order to truly make that statement, one would have to implement a given scheme on both meshes and then compare. My experience on the cubed-sphere mesh is that it.

\item [Response]:
Agreed. Our test cases on a plane will highlight errors due to non-orthogonality and due to discontinuities at cube edges but cannot help to describe errors at cube corners. Some sentences have been added to the introduction:
\begin{quote}
We assume that a large fraction of the numerical errors associated with terrain following co-ordinates and the cubed-sphere grid are due to the mesh non-orthoganlity and distortion. However there will also be errors due to cubed-sphere corners and inaccurate representation of spherical geometry. These are not addressed here. 
\end{quote}
and
\begin{quote}
These test cases will mimic the problems encountered at cubed-sphere edges but will not give any information about errors that might occur at cubed-sphere corners. 
\end{quote}
A similar caveat has been inserted in the second paragraph of the abstract, the first paragraph of section 3 and in the third paragraph of the conclusions. 

\item [Comment 3]: Most importantly, to say the newly proposed dimensionally split scheme is accurate compared to multi-dimensional schemes, I believe a better comparison point is needed. Users select the scheme that gives the best results with the fewest resources. I believe the authors need to include a time-explicit semi-discrete (method of lines) multi-dimensional scheme with cubic interpolants using smaller time steps to serve as a comparison point for this scheme. I know the focus is on large time steps, but it's necessary to consider if one can, in fact, get more accuracy for a given amount of work by simply using the status quo with smaller time steps. 

\item [Response]: 
We now compare the dimensionally split scheme with the multi-dimensional scheme when the multi-dimensional scheme is run with explicit and implicit time-stepping. New results are described in section 3. The errors of the multi-dimensional scheme (which is now described in full by \cite{SWMD17}) are insensitive to the choice of time-stepping type (implicit or explicit) for Courant numbers smaller than two. These comparisons have also revealed that the implicit solve is a small part of the total cost of the advection. This is likely because the linear equation solvers in OpenFOAM are well optimised, our test cases are run in serial and there are cache misses associated with the explicit part of the advection which uses extended stencils. The multi-dimensional scheme uses cubic interpolants but is not overall a third-order accurate. The multi-dimensional scheme is designed for arbitrary meshes. In order to achieve third-order accuracy on arbitrary meshes either multi-moments or Gauss points within each cell are needed. This would increase the complexity and cost significantly. The multi-dimensional scheme run explicitly has similar cost and complexity to the dimensionally split scheme. 

\end{description}

\section*{Responses to reviewer 1 pdf annotations}

Page and line numbers refer to the original annotated document.

\begin{enumerate}
\item [Page 1, Line 41]: Might want to say ``distorted, non-orthogonal meshes'' to make it's clear that you are indeed directly testing non-orthogonality.

\item [Response]: Changed as suggested.

\item [Page 2, Line 4]: quasi-uniform resolution doesn't help better exploit modern architectures; they're largely orthogonal concepts. Quasi-uniform resolution just improves the global time step restriction.

\item [Response]: Quasi-uniform resolution indeed help better exploit modern architectures. The lat-lon grid leads to convergence of resolution toward poles and restricts the time stepping. These problems are usually solved by semi-implicit semi-Lagrangian schemes. The modern computers use increasing number of cores and the efficiency is limited by the scalability. Implicit solvers and semi-Lagrangian advection result in massive data communications on lat-lon grids \citep{ST12}, which limits the scalability around poles. 

I know what you mean but, in reality, the modern computer architectures really are driving the need for quasi-uniform meshes and this is well documented in the references that we cite. We have changed the sentence to:
\begin{quote}
Weather and climate models are being developed on quasi-uniform meshes in order to equidistribute and reduce the storage needed for the domain of dependence of the model degrees of freedom and hence better exploit modern computers \cite[eg][]{WTC12,LUJ+14,ST12,SG11,KNK15}.
\end{quote}

\item [Page 2, Line 27]: I feel the strong need for a caveat here. I appreciate the desire to create a simpler test case not involving spherical geometry, because the cubed-sphere really is easy to mess up. However, my personal experience is that dimensional splitting  on the cubed-sphere incurs significantly more error compared to dimensional splitting on a plane. This could potentially be because spherical geometry coupled with non-orthogonality incurs more splitting error than non-orthogonality alone. Without directly confirming the applicability of this framework to the cubed-sphere by correlating error measures between them, I think the authors need to qualify more clearly that this applicability remains a *hypothesis.*

\item [Response]: Agreed. We have added the following to the end of this paragraph:
\begin{quote}
We assume that a large fraction of the numerical errors associated with terrain following co-ordinates and the cubed-sphere grid are due to the mesh non-orthoganlity and distortion. However there will also be errors due to cubed-sphere corners and inaccurate representation of spherical geometry. There are not addressed here. 
\end{quote}

\item [Page 2, Line 40]: Why would remapping otherwise be needed? This sounds like Yin-Yang kind of stuff. If so, please explicitly mention that.

\item [Response]: I was thinking about schemes like SLICE and CSLAM here but on reflection it is not best to discuss these schemes at this point so this comment has been removed. 

\item [Page 2, Line 43]: temporally second-order accurate

\item [Response]: changed as suggested

\item [Page 2, Line 45]: These guys \cite[]{PL07} appear to be using a sort of predictor-corrector approach where one predicts in the $x$-direction to $\Delta t/2$, then corrects with $y$ over the time-step. Then they swap the dimensions and average the final result. Their errors are certainly more uniform over time, but they are larger (using $90\times 90$ DOFs per panel) than \cite{GNQ14} (who used $80\times 80$ DOFs per panel).

\item [Page 2, Line 46]: The scheme \cite[]{KNK15} in this paper isn't really dimensionally split, only the *reconstruction* is. \cite{KNK15} use dimensionally split sweeps to create a genuinely multi-dimensional interpolant (minus a few of the cross terms), and then they evolve in a genuinely multi-dimensional way. This is an important distinction because (1) they do not incur time error in the same way as truly split schemes and (2) they include a diagonal sweep for some of the cross terms, whereas truly split schemes generally do not.

\item [Response]: These sentences have been changed to:
\begin{quote}
Dimensionally split schemes have been found to give good accuracy on non-orthogonal meshes such as the cubed-sphere if special treatment is applied over cube edges to improve accuracy: \cite{PL07} use the average of two one-sided schemes at cube edges. \cite{KNK15} create ghost cells outside each cube panel boundary and achieve accuracy between second and fourth order. \cite{KNK15} use a multi-dimensional transport scheme with dimensionally split reconstructions. \cite[eg]{GNQ14} report error growth around cube edges using Strang splitting. 
\end{quote}

\item [Page 2, Line 46]: There's another paper that actually does do dimensional splitting on the cubed-sphere:

\url{http://journals.ametsoc.org/doi/abs/10.1175/MWR-D-13-00048.1}

And they incur very large L-inf errors when the advected shape passes cubed-sphere corners. Their errors are smaller with their Strang carry-over scheme than one would experience with an alternating Strang splitting, but the errors are still quite high. Also, these errors at the cubed-sphere corners reduce significantly at smaller time steps, as expected, but it puts a bit of a damper on the desire to use larger time steps.

\item [Response]: Thank-you for pointing out this reference. Included as described above. 

%On this paper, on page 460: 
%``It is difficult to design a numerical scheme that preserves a constant field in the dimensional splitting framework. It is our ongoing work to design a nonsplitting SLDG scheme that preserves the constant field when the velocity field is nondivergent.'' 
%Is this because of the filter or the splitting? How is the performance of the scheme under 1-D divergnet flow, because the 2-D non-divergent field could lead to 1-D divergent flow in each dimension.

%On page 464, ``It is clear that the second order of convergence is observed, which comes from the splitting error.''
%I believe in solid body rotation test case on Cartesian grid, the splitting error is zero in COSMIC splitting if the splitting scheme is 2nd order in time. So I think the splitting scheme is not comparable to the COSMIC splitting.

%On page 469, ``This is an indication that such error comes from the dimensional splitting, and there exists a certain symmetry property for different dimensional-splitting orderings. Such symmetry property, together with the symmetry of the cosine bell profile, may contribute to the dropping of the $l_\infty$ error after the cosine bell passed the corner.''
%If this is the reason for the large error, COSMIC splitting may give a better results as COSMIC splitting get a symmetric results by averaging two splitting scheme.

\item [Page 2, Line 55]: It's important to note that all \cite{SLF+02}  did was compress all of the errors to the lowest few levels, where the metric Jacobian determinants vary even more.

\item [Response] Agreed. Added the phrase ``away from the ground''.

\item [Page 2, Line 57]: Remapping is where the error in incurred here. The advantage is that one can do that less often with floating layers.

\item [Response]: There are of course errors from remapping but we are unaware that these errors are worse where the grid is highly distorted over orography. Therefore we have left the text as it is.

\item [Page 3, Line 4]: \cite{UN14} performed a multi-dimensional FFSL method with hyperdiffusion out to a CFL of roughly 2.5

\item [Response]. Thank-you. We have added:
\begin{quote}
a notable exception being \cite{UN14}, whose FFSL semi-Lagrangian method is stable for Courant numbers up to around 2.5.
\end{quote}

\item [Page 3, Line 7]: I'm not so sure this is true. The reconstruction cost would be the same regardless. It's the integration cost that would be proportional with CFL. Of course, as the CFL increases, integration increasingly becomes more and more of the model cost. So this is asymptotically true, but not necessarily true for more moderate CFL numbers.

\item [Response]: Agreed. Added ``integration cost''

\item [Page 3, Line 14]: A third disadvantage is that they require MPI\_all\_reduce for the linear solves, which is increasingly prohibitive in its expense on large computers.

\item [Response]: Added ``the parallelization cost of linear solvers''.

\item [Page 3, Line 14]: I believe this has been solved with SSP implicit RK and SSP DIRK schemes, though, right?

\item [Response]: I don't think so because monotonic schemes are non-linear. We have left this as ``the \emph{difficulty} of achieving monotonicity'' so as not to exclude the possibility. 

\item [Page 3, Line 31]: It's worth mentioning that significantly less special treatment is needed for element-based schemes (e.g., Galerkin) because no halo is needed. They merely transform vector quantities at element edges that coincide with panel boundaries.

\item [Response]: Actually we wanted to talk about testing advection schemes here rather than properties of advection schemes so the sentence has been replaced by:
\begin{quote}
The theoretical properties of dimensionally split advection schemes are often tested on uniform, orthogonal meshes \cite[eg][]{LLM96} which is inadequate for a scheme that will eventually be used on a cubed-sphere.
\end{quote}

\item [Page 3, Line 45]: What do you mean by maximally? Do you mean ``highly'', or is something being optimized here?

\item [Response]: Replaced by:
\begin{quote}
without smoothing of terrain following layers
\end{quote}

\item [Page 5, Line 1]: Thanks for mentioning this. Do you think using quadrature in time would be worthwhile? It'd have to be separate over each individual cell because it requires continuity, so it'd be pretty expensive

\item [Response]: We have added:
\begin{quote}
The next step would be to use a second-order predictor-corrector departure point calculation \cite[eg][]{MDW+10}.
\end{quote}

\item [Page 5, Line 27]: Is this done for sake of expense, since advective SL schemes only need one upwind evaluation, and not an integral; or are the other reasons for this?

\item [Response]: COSMIC splitting is defined as using both advective and conservative advection operators in order to maintain stability, conservation, preservation of constants and second-order accuracy. This is already described so we haven't changed the text here.

\item [Page 6, Line 15]: I find it confusing that you show a hexagonal mesh here and yet are only using rectangular meshes in the study.

\item [Response]: Removed the hexagonal mesh.

\item [Page 6, Line 40]: Can you use $a_1$, $a_2$, etc here, because ``$i$'' has a meaning that could otherwise be confusing.

\item [Response]: Yes, done.

\item [Page 8, Line 57]: It seems to me that the implicit scheme just isn't a very good comparison point. I think it would be more helpful to see a smaller time step, fully explicit, multi-dimensional comparison point.

If the purpose of this paper were solely to introduce new test cases, I think the existing methods would be OK. But since the purpose is to also introduce a new scheme, I think having a better comparison point is important. Since for many models, the state of the art is using smaller-time-step explicit schemes, it's important to see how the accuracy compares to that.

\item [Response]: See response to comment 3 above. 

\item [Page 10, Line 55]: Have you thought about using a more accurate implicit time stepping than CN? It would probably improve the phase errors.

\item [Response]: Good idea. Simpler than improving the spatial accuracy of the multi-dimensional scheme. However improving the spatial accuracy is a higher priority since for Courant numbers less than one, the multi-dimensional scheme errors are insensitive to time-step. No change made to manuscript.

\item [Page 13, Line 1]: This is really a comparative statement, and I think the baseline is fairly poor.

\item [Response]: We agree that the baseline could be better. However the two schemes do have similar complexity and cost. The final sentence has been changed to:
\begin{quote}
In contrast, the multi-dimensional scheme is insensitive to the orthogonality, converges with second-order \textcolor{blue}{as expected} and suffers from phase errors at large Courant numbers \textcolor{blue}{due to the use of Crank-Nicolson implicit time-stepping}.
\end{quote}

\item [Page 13, Line 19]: Does it really ``maximize'' the non-orthogonality or just make for a ``highly'' non-orthogonal test case?

\item [Response]: We have changed this to:
\begin{quote}
maximises the non-orthogonality that can be introduced by terrain following coordinates.
\end{quote}
Smoother terrain following coordinates are more orthogonal but less uniform near the ground. 

\item [Page 15, Line 36]: I think it's worth mentioning why we dimensionally split off the vertical: the high aspect ratio present in most weather/climate models.

\item [Response]: We add explanation at the start of this section as suggested.

\item [Page 22, Line 41]: I'd remove words like ``very'' and ``extremely.'' They aren't well-defined, and it's better to use comparative language with a clear baseline.

\item [Response]: Agreed. We removed the words ``very'' and "extremely'' in the following locations:
\begin{itemize}
\item One instance of  ``very'' in the abstract.
\item One instance of  ``very'' in the conclusions.
\item One instance of  ``extremely'' in the conclusions.
\end{itemize}
Also caveats have been added to the first paragraph of the conclusions explaining the limitations of the comparison.

\end{enumerate}

\section*{Responses to reviewer 2 (Lucas Harris)}

\begin{enumerate}
\item [Major comment 1]: The first questions that come to my mind when reading this paper are, ``why another advection scheme?'' and ``why another advection test case?'' given the multitude of existing schemes and tests. The demonstration of the good accuracy of standard dimensionally-split FFSL schemes is certainly useful, especially considering the well-publicized shortcomings of such methods. However, I am concerned that the comparison is somewhat limited by the fact that the multi-dimensional scheme is of lower-order than the dimensionally-split scheme. How can the lower order of accuracy of the multi-dimensional scheme justified?

\item [Response]: 
Our initial motivation was to demonstrate the large errors associated with dimensionally split schemes on non-orthogonal meshes with discontinuities in co-ordinate transforms which we thought had not previously been demonstrated in the literature. We could not find these large errors despite comparing results on orthogonal and non-orthogonal meshes. 

Accuracy higher than second order is not straightforward for a multi-dimensional scheme suitable for arbitrary meshes. The complexity and cost of the multi-dimensional scheme is similar to that of the dimensionally split scheme. However, in fact, neither scheme has accuracy higher than second order although they both have aspects that are third-order accurate. The dimensionally split scheme is third or forth order accurate in one dimension but the COSMIC splitting reduces the temporal order of accuracy even on an orthogonal grid. We thought that discontinuities in the coordinate system would reduce the order of accuracy to first but it appears to be second order. The multi-dimensional scheme uses cubic interpolations but since we use the cell centre as the Gauss point for each cell, the multi-dimensional scheme is limited to second order accuracy. We have added this text to the end of the first paragraph of section 2 and similar wording to the beginning of the conclusions:
\begin{quote}
The dimensionally split scheme is third-order accurate in one dimension but the COSMIC splitting reduces the temporal order to two. The multi-dimensional scheme uses cubic interpolations but uses cell centoids as Gauss points so the order of accuracy is limited to two. When run explicitly, the cost of both schemes is similar (see section 3.4). These schemes are therefore considered suitable for comparison. 
\end{quote}

\item [Major comment 2]: It is also questionable how useful it is to compare advection schemes without any form of monotonicity preservation, which is a necessity for modern solvers. The effect of this is most visible in Figure 4, where the more accurate dimensionally-split scheme has more grid-scale noise than the lower-order multi-dimensional scheme, likely because of the larger implicit diffusion in the lower-order scheme. A proper monotone FFSL scheme would greatly reduce these ripples with only a modest introduction of diffusive errors.

\item [Response]: We agree that advection schemes without monotonicity preservation are of limited use. While some of the dimensionally split scheme results have oscillations, they are still more accurate than the smooth results from the multi-dimensional scheme. Therefore including a good monotone limiter shouldn't change the conclusions of this paper. I am afraid that we no longer have the resources to implement a good monotone limiter for the dimensionally split scheme. We feel that these points are already covered sufficiently in section 3.1:
\begin{quote}
The dimensionally split scheme outperforms the multi-dimensional scheme on both meshes due to the higher-order accuracy of the split scheme. The dimensionally split scheme introduces a small error at 300 seconds where the tracer goes through the change in direction of the mesh which would be ameliorated if we were using monotonicity constraints. The second-order, multi-dimensional scheme shows phase lag but errors are almost entirely insensitive the mesh distortions. 
\end{quote}

\item [Minor Comment 1]: Given that a fundamentally new scheme is not presented here, a better title might be something like, ``Comparison of dimensionally-split and multi-dimensional atmospheric transport schemes for long time-steps''

\item [Response]: Agreed, thanks.

\item [Minor Comment 2]: Several references to Lin and Rood (1996) should be to Putman and Lin (2007), since only PL07 uses the cubed-sphere. LR96 is designed for cartesian and lat-lon grids. Also, only LR96 uses FFSL methods, while PL07 has dropped the semi-Lagrangian extension (which was most useful near the poles of lat-lon grids). 

\item [Response]: Thanks, changed.

\item [Minor Comment 3]: Figure 8: Is the variation in grid spacing done while keeping the width of the orography constant? 

\item [Response]: Yes. This is now clarified in section 3.2:
\begin{quote}
The width and height of the orography is kept constant.
\end{quote}

\item [Minor Comment 4]: Section 3.4: The discussion of computational cost of the algorithms is a very useful addition to the paper. The brief discussion of 3D extensions is also very useful.

\item [Response]: Thank-you. Following a request from reviewer 1, we also now include a comparison of wall clock time for the multi-dimensional scheme run implicitly and explicitly in serial. 
\end{enumerate}


\bibliographystyle{abbrvnat}
\bibliography{numerics}

\end{document}
