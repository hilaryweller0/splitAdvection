The dimensionally split scheme decribed in this paper is known as Conservative Operator Splitting for Multidimensions with Inherently Constancy (COSMIC) splitting \citep{Leonard1996} or flux-form semi-Lagrangian \citep{Lin1996}. This dimension splitting scheme allows one-dimensional piecewise parabolic method (PPM) \citep{Colella1984} combined with a long time steps technique \citep{Leonard1995} to be extended to two-dimensions.
\subsubsection{Long time-step permitting PPM} 
\label{sec:PPM}
Both PPM and the long time steps technique use subgrid cell reconstruction. Although they are defined for varying grid spacing, we will only describe them for uniform grid interval, $\Delta x$. The scheme is described on a domain with length $L$, which is divided into $n$ grid points. The position of $j$th grid is $x_j = j\Delta x$ and $j = 0, 1, 2, \cdots, n-1$ and the corresponding tracer is $\phi_j$. Subgrid cell reconstruction uses the space average mass within the $j$th grid box at time $t_n$, which is defined as: 
\begin{equation} \label{eq:2.1} 
\phi^n_j = \frac{1}{\Delta x} \int^{x_{j+\frac{1}{2}}}_{x_{j-\frac{1}{2}}} \phi (x,t)dx
\end{equation}
where $x_{j-\frac{1}{2}}$ is the midpoint between $x_{j-1}$ and $x_{j}$ and is also the boundary of each grid point. Thus, $x_{j-\frac{1}{2}} = x_j - \frac{\Delta x}{2}$.
\begin{figure}
% \vspace{-20pt}
\centering
\includegraphics[scale=0.4]{YumengsGraphics/nirvana.pdf}
% \vspace{-20pt}
\caption{The 1D long time permitting technique. The mass between $x_{j,L}$ and $x_{j-c_N-1,R}$ can be calculated using equation \ref{eq:2.3}, and the mass between $x_{j-c_N-1,R}$ and $x_{j-c}$ should be calculated using PPM \label{fig:2.1}}
\end{figure}

The long time step permitting technique is illustrated in figure \ref{fig:2.1}. Similar to semi-Lagrangian, the long time step permitting feature is achieved by tracing back to the departure point and the time step is restricted by the deformational Courant number $c_d$. \cite{Leonard1995} shows that the flux at the left boundary $x_{j-\frac{1}{2}}$ can be obtained using the average mass between $x_{j-\frac{1}{2}}$ and $x_{j-\frac{1}{2}}-c\Delta x$.

It is convinient to divide the Courant number, $c$, into integer part, $c_N$, and decimal part (or remnant Courant number), $c_r$ \citep{Leonard1995}. In this way, the CFL restriction is removed, because the remnant Courant number is always less than one. The flux equation becomes:
\begin{eqnarray} \label{eq:2.2}
c\cdot \Delta x \phi_{j-\frac{1}{2}} = M_{c_r,R,j-1} + M_{j-\frac{1}{2}}-M_{j-c_N-\frac{1}{2}} & \text{c}>0 &\nonumber \\
c\cdot \Delta x \phi_{j-\frac{1}{2}} = -M_{c_r,R,j-1} +M_{j+c_N-\frac{1}{2}}- M_{j-\frac{1}{2}} & \text{c}<0 & \\
c\cdot \Delta x \phi_{j-\frac{1}{2}} = 0 & \text{otherwise} \nonumber 
\end{eqnarray}
where $M_{j-\frac{1}{2}}$ is the cumulative mass from the start point to the $j$th left zone edges defined as \citep{Colella1984}:
\begin{equation} \label{eq:2.3}
M_{j-\frac{1}{2}} = \sum_{k < j} \phi_{k}\Delta x
\end{equation}

The $M_{c_r,R,j-1}$ in equation \ref{eq:2.2} is calculated by PPM, which assumes that the profile at each interval is a parabola \citep{Colella1984}. The parabolic equation can be achieved using $\phi(\xi = 0) = \phi_L$, $\phi(\xi = 1) = \phi_R$ and $ \int^{1}_{0} \phi (\xi) d\xi = \phi_j  $, where $\phi_L$ and $\phi_R$ is obtained by a quartic polynomial mass interpolation. The mass flux thus can be calculated in terms of the tracer at the boundaries of each interval.

Therefore, for westerly flows, the formula for next time step $\phi^{n+1}_j$ is: 
\begin{equation} \label{eq:2.12}
\phi^{n+1}_j = \phi^n_j + \frac{M^n_{in,j} - M^n_{out,j}}{\Delta x} = \phi^n_j + c(\phi_{j-\frac{1}{2}} - \phi_{j+\frac{1}{2}})%\frac{M^n_{R,j} - M^n_{R,j+1}}{\Delta x}
\end{equation}
where the subscript means the inflow and outflow of each cell.

\subsubsection{COSMIC Splitting}
\label{sec:COSMIC}
COSMIC splitting is an operator splitting scheme, which is used so that PPM can be implemented in two dimensions. Three of the properties expected of multidimensional advection schemes are: stability, conservation and constancy preservation \citep{Lin1996}.
\begin{figure}
\vspace{-10pt}
\centering
\includegraphics[scale=0.4]{YumengsGraphics/conservative_form.pdf}
\vspace{-10pt}
\caption{The position of each flux and the position of $\phi$, the subscript w/e/n/s denote the direction of the flux relative to the position of $\phi$ \label{fig:2.2}}
\end{figure}
Operator splitting can use conservative-form or advective-form operators. Use of conservative-form operator with operator splitting leads to inherently mass conservation but cannot satisfy the constancy condition \citep{Leonard1996}. The conservative-form operator in $x$ and $y$ direction is:
\begin{eqnarray}  \label{eq:2.13}
&X_{\scriptscriptstyle C}(\phi ) = c_{w}\phi_w(\phi )-c_e\phi_e(\phi ) 
&Y_{\scriptscriptstyle C}(\phi ) = c_{n}\phi_n(\phi )-c_s\phi_s(\phi )
\end{eqnarray}
where $c$ is the Courant number and the subscripts represent the direction of the boundary of each grid and w/e/n/s denote the compass direction relative to $\phi$, as shown in figure \ref{fig:2.2}. The subscript $C$ represents the conservative-form operator.

Advective-form operator splitting is another way to treat the advection scheme. Unlike the conservative-form, the advective-form splitting is not inherently conservative but can satisfy constancy condition. The advective-form operator \citep{Leonard1996} can be defined as:
\begin{eqnarray} \label{eq:2.16}
&X_{\scriptscriptstyle A}(\phi ) = c^{cell}(\phi_w(\phi )-\phi_e(\phi )) 
&Y_{\scriptscriptstyle A}(\phi ) = c^{cell}(\phi_n(\phi )-\phi_s(\phi ))
\end{eqnarray}
where $c^{cell}$ is the cell centred Courant number, or the transverse velocity and the subscript $A$ denote advective-form operator. The cell centred Courant number should be ideally the average of two faces. However, \citep{Lin1996} pointed out that this could generate spurious flux, so the transverse velocity should be chosen as the upwind velocity. If the transverse velocity is in the opposite directions at two faces, the cell centred velocity should set to zero.

In order to satisfy both conservation and constancy condition, COSMIC splitting \citep{Leonard1996} combines the conservative-form and advective-form update. Thus, the COSMIC splitting can be written as:
\begin{equation} \label{eq:2.19}
\phi^{n+1} = \phi^n+X_{\scriptscriptstyle C}(\frac{1}{2}(\phi^n + \phi^n _{\scriptscriptstyle AY}) +Y_{\scriptscriptstyle C}(\frac{1}{2}(\phi^n + \phi^n _{\scriptscriptstyle AX})
\end{equation}
According to equation \ref{eq:2.19}, conservation is preserved by the outer conservative-form operator while constancy is maintained by the advective-form update. COSMIC splitting can be interpreted as the mass flux into control-volume cell.% \cite{Lauritzen2007} showed that part of the mass comes from the departure cell, which is called local contribution  while there are nonlocal contributions if the Courant number is larger than one and the $\phi_{CX} \neq \phi_{AX}$.
